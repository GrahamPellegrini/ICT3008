\documentclass{article}

\title{Case Study 1 – Career, Honesty and Gender}
\author{Graham Pellegrini}
\date{\today}

\begin{document}

\maketitle

\section{Question 1}
Ben is a pacifist engineer who is concealing a potentially valuable innovation from his employer as it could be used for military purposes, despite having signed an agreement that all his work-related inventions belong to the company.\\

\textbf{Do you think it is ethically right for Ben to conceal his idea from his employers?}

\subsection{Yes}

\begin{itemize}
    \item Personal Morals - Ben is a pacifist and has the right to adhere to his own moral beliefs. He should not be compelled to compromise his principles for the benefit of the company.
    \item Public Interest - Military innovation can be argued to not serve the public interest. Such innovations are often used to threaten or harm opposing forces and can be a primary cause of war, as seen historically.
    \item Idea vs. Innovation - The distinction between ideas and innovations is often unclear. Since Ben has not further developed the idea, it remains an idea rather than an innovation. Therefore, he is not violating the agreement.
    \item Employer's Prior Knowledge - The employer is aware of Ben's pacifist beliefs and should not expect him to produce results that contradict his principles.
\end{itemize}

\subsection{No}

\begin{itemize}
    \item Contractual Obligations - Ben has signed a legal agreement. Not developing or reporting the idea is a breach of contract. He does not have personal ownership of the idea, as it was pursued during the course of his employment.
    \item Subjective Interest - Not all military innovations are used for harm. Some are used for defense and protection. Ben’s perception is not definitive and should not justify concealing the development of the idea.
    \item Transparency - As an employee, Ben is obliged to be transparent with his employer. He should not conceal any information that could be of value to the company. He can always discuss his concerns with the employer and agree not to work further on the idea.
    \item Other Potentially Valuable Uses - The idea may have other valuable applications. For example, ultrasonic technology can be used in the medical field. By concealing the idea, Ben could potentially be stagnating innovation in other sectors as well.
\end{itemize}

\section{Question 2}
Harrison Brown was a scientist who worked on the Manhattan Project but later advocated for the ethical responsibility of scientists. He wrote books urging global control of nuclear weapons and awareness of technological risks. His work influenced policymakers to address the intersection of science, technology, and society.\\

\textbf{You are a computer engineer and you are assigned to the Manhattan Project. Do you think it is ethical to work on this project?}

\subsection{Yes}
\begin{itemize}
    \item National Security - Considering the historical timeline, the project’s efforts were ethically aimed at ending World War II, combating the Axis powers, and preventing further loss of life.
    \item Scientific and Technological Advancements - The project led to major breakthroughs in scientific fields, which later resulted in positive outcomes.
    \item Uncertainty - At the time of the project, the full extent of the damage caused by the atomic bomb was not known. The engineers and scientists could not have predicted the long-term effects of the bomb, and their ethical judgment was based on the information available at the time.
    \item Obligation to One’s Country - Engineers and scientists were employed by the government and had a duty to contribute to their nation’s defense.
\end{itemize}

\subsection{No}
\begin{itemize}
    \item Ethical Responsibility - The project led to the deaths of thousands of innocent civilians. The engineers and scientists had a moral obligation to consider the ethical implications of their work.
    \item Long-Term Consequences - The atomic bomb had long-term effects on the environment and human health. The engineers and scientists should have considered the potential consequences of their work.
    \item Global Responsibility - The engineers and scientists had a responsibility to consider the global implications of their work. The atomic bomb had far-reaching effects on the world, and the engineers and scientists should have considered the impact of their work on the global community.
    \item Alternative Solutions - There were alternative solutions to ending the war that did not involve the use of the atomic bomb. The engineers and scientists should have considered these alternatives before proceeding with the project.
\end{itemize}

\section{Question 3}
Women remain underrepresented in U.S. engineering and science schools, making up about 20\% of undergraduates and less than 5\% of full professors. Gender biases, or gender schemas, unconsciously favor men, leading to women being systematically underrated in evaluations, reference letters, and leadership roles. Despite good intentions, these biases accumulate over time, disadvantaging women. Universities like Michigan are addressing this through workshops and research, emphasizing that active efforts—not just goodwill—are needed to promote fairness and inclusion.\\

\textbf{Positive discrimination in favor of women is often proposed as a measure to address gender imbalance. What are the advantages and disadvantages of such measures?}

\subsubsection{Agreement with Positive Discrimination}
\begin{itemize}
    \item Increased Diversity - Having a diverse workforce can lead to increased creativity and innovation. Men and women tend to have different perspectives and ideas due to differences in emotional intelligence, general knowledge, and experiences.
    \item Breaking the Cycle - Positive discrimination can help break the cycle of bias. As future potential women engineers see greater representation, they will not feel discouraged from pursuing a career in engineering.
    \item Creates Role Models - Role models often share similar traits with their followers. Having more women in leadership roles can inspire other women to strive for similar positions.
    \item Compensation for Past Discrimination - Positive discrimination can help compensate for past discrimination and create a more level playing field between genders.
\end{itemize}

\subsubsection{Disagreement with Positive Discrimination}
\begin{itemize}
    \item Pressure on Women in Roles - Women placed at the forefront of positive discrimination or in leadership roles will be under pressure to perform. This pressure can lead to stress and burnout, potentially having a negative impact.
    \item Reverse Discrimination - Positive discrimination can lead to reverse discrimination against men. Catering to one group often places another at a disadvantage. This can lead to resentment and further discrimination.
    \item Doesn't Address Root Cause - Positive discrimination does not address the root cause of gender imbalance. It does not tackle the underlying biases and stereotypes that contribute to the issue.
    \item Tokenism - Women in the workforce may feel like they are only there to fill a quota rather than being valued for their skills and abilities. This can lead to a lack of confidence and motivation.
\end{itemize}

\section{Question 4}
\textbf{You are the CEO of a local computer engineering company that employs 150 people. What measures and policies will you put in place to move towards gender balance?}
\subsection{Ideas}
\begin{itemize}
    \item Blind Resume Screening - By reviewing resumes without names, gender, or any other identifying information, the company can ensure that hiring decisions are based on qualifications and experience rather than potential biases.
    \item Mentorship Programs - Establishing mentorship programs where senior employees, both male and female, guide and support female employees in career advancement.
    \item Flexible Work Policies - Providing flexible work hours, parental leave, and remote work options to accommodate diverse needs.
    \item Targeted Recruitment - Actively reaching out to women through job fairs, partnerships with universities, and industry organizations.
    \item Equal Pay Audits - Regularly assessing salary structures to ensure gender pay equity within the company.
    \item Inclusive Workplace Culture - Conducting training on unconscious bias, fostering an inclusive culture, and encouraging diverse leadership representation.
    \item Promotion Transparency - Setting clear criteria for promotions and leadership roles to prevent unconscious bias in career advancement.
\end{itemize}

\subsection{Arguments Against Ideas}
\begin{itemize}
    \item Blind Resume Screening - While effective in initial stages, interviews and networking still allow biases to influence decisions.
    \item Mentorship Programs - Some may argue that structured programs could create resentment among employees who do not receive mentorship.
    \item Flexible Work Policies - There could be concerns about fairness if some employees perceive that flexible options benefit only a specific group.
    \item Targeted Recruitment - This might be viewed as preferential treatment and could lead to pushback from other employees.
    \item Equal Pay Audits - Resistance might arise due to the cost and effort of conducting frequent audits.
    \item Inclusive Workplace Culture - Changing workplace culture takes time, and some employees may resist diversity initiatives.
    \item Promotion Transparency - Ensuring transparency while maintaining confidentiality and managerial discretion can be challenging.
\end{itemize}
\end{document}

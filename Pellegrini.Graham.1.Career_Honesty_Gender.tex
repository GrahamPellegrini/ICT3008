\documentclass[12pt]{article}

\usepackage{setspace}
\onehalfspacing % Increase spacing
\usepackage[a4paper, margin=1in]{geometry} % Adjust margins

\title{Case Study 1 – Career, Honesty and Gender}
\author{Graham Pellegrini}
\date{\today}

\begin{document}

\maketitle

\section{Question 1}
\textbf{Is it ethically right for Ben to conceal his idea from his employers? (Group A - Yes)}

\textit{Group A - No}  
\textit{Group B - Yes}\\

\noindent The debate began with Group A asserting:
{\fontsize{11pt}{13pt}\selectfont
\begin{itemize}
\setlength{\itemsep}{4pt} % Adds spacing between items
    \item Ben has the right to adhere to his moral beliefs.
    \item The idea is not fully developed, raising ethical and contractual considerations.
    \item From a utilitarian perspective, concealing the idea benefits more lives.
\end{itemize}
}

\noindent Group B countered by arguing:
{\fontsize{11pt}{13pt}\selectfont
\begin{itemize}
\setlength{\itemsep}{4pt} 
    \item Concealing the idea breaches transparency and employer trust.
    \item Decision-making on such dilemmas should be left to legislators, not employees.
    \item Sharing the idea responsibly could prevent misuse.
\end{itemize}
}

\noindent Group A reinforced that the idea is not concrete and does not deprive the company of valuable information. They emphasized that Ben's pacifist stance was known when he was hired, so expectations should align with his beliefs.\\

\noindent Group B introduced counterarguments:
{\fontsize{11pt}{13pt}\selectfont
\begin{itemize}
\setlength{\itemsep}{4pt} 
    \item Ben benefits from company resources and owes contribution in return.
    \item Military advancements can protect soldiers and deter threats.
    \item The idea could have broader scientific applications beyond military use.
\end{itemize}
}

\noindent Group A responded:
{\fontsize{11pt}{13pt}\selectfont
\begin{itemize}
\setlength{\itemsep}{4pt}
    \item The contract only covers developed ideas, and this one remains undeveloped.
    \item A perpetual arms race could result in catastrophic consequences.
    \item Forcing Ben to work against his beliefs could be unproductive and detrimental to morale.
\end{itemize}
}

\noindent Further counterarguments from Group B emphasized:
{\fontsize{11pt}{13pt}\selectfont
\begin{itemize}
\setlength{\itemsep}{4pt} 
    \item Decisions of this magnitude should be made transparently within a committee.
    \item Concealment hinders company growth and potential partnerships.
    \item Military advancements often lead to public technological benefits.
    \item If Ben does not develop the idea, someone else will, potentially without ethical considerations.
\end{itemize}
}

\noindent Group A concluded by highlighting:
{\fontsize{11pt}{13pt}\selectfont
\begin{itemize}
\setlength{\itemsep}{4pt} 
    \item The risks of escalating military capabilities, especially in the current political climate.
    \item The company is not in the military sector, so Ben did not sign up for this work.
    \item If technological advancements are inevitable, Ben still has the moral right to abstain.
\end{itemize}
}

\section{Question 2}
\textbf{You are a computer engineer and you are assigned on the Manhattan Project. Do you think it is ethical to work on this project? (Group A No)}
\textit {Group A - No}
\textit {Group B - Yes}
42:27
\end{document}
\documentclass[12pt]{article}

\usepackage{setspace}
\onehalfspacing % Increase spacing
\usepackage[a4paper, margin=1in]{geometry} % Adjust margins

\title{Case Study 1 – Career, Honesty and Gender}
\author{Graham Pellegrini}
\date{\today}

\begin{document}

\maketitle

\section{Question 1}
\textbf{Is it ethically right for Ben to conceal his idea from his employers? (Group A - Yes)}

\textit{Group A - No}  
\textit{Group B - Yes}\\

\noindent The debate began with Group A asserting:
{\fontsize{11pt}{13pt}\selectfont
\begin{itemize}
\setlength{\itemsep}{4pt} % Adds spacing between items
    \item Ben has the right to adhere to his moral beliefs.
    \item The idea is not fully developed, raising ethical and contractual considerations.
    \item From a utilitarian perspective, concealing the idea benefits more lives.
\end{itemize}
}

\noindent Group B countered by arguing:
{\fontsize{11pt}{13pt}\selectfont
\begin{itemize}
\setlength{\itemsep}{4pt} 
    \item Concealing the idea breaches transparency and employer trust.
    \item Decision-making on such dilemmas should be left to legislators, not employees.
    \item Sharing the idea responsibly could prevent misuse.
\end{itemize}
}

\noindent Group A reinforced that the idea is not concrete and does not deprive the company of valuable information. They emphasized that Ben's pacifist stance was known when he was hired, so expectations should align with his beliefs.\\

\noindent Group B introduced counterarguments:
{\fontsize{11pt}{13pt}\selectfont
\begin{itemize}
\setlength{\itemsep}{4pt} 
    \item Ben benefits from company resources and owes contribution in return.
    \item Military advancements can protect soldiers and deter threats.
    \item The idea could have broader scientific applications beyond military use.
\end{itemize}
}

\noindent Group A responded:
{\fontsize{11pt}{13pt}\selectfont
\begin{itemize}
\setlength{\itemsep}{4pt}
    \item The contract only covers developed ideas, and this one remains undeveloped.
    \item A perpetual arms race could result in catastrophic consequences.
    \item Forcing Ben to work against his beliefs could be unproductive and detrimental to morale.
\end{itemize}
}

\noindent Further counterarguments from Group B emphasized:
{\fontsize{11pt}{13pt}\selectfont
\begin{itemize}
\setlength{\itemsep}{4pt} 
    \item Decisions of this magnitude should be made transparently within a committee.
    \item Concealment hinders company growth and potential partnerships.
    \item Military advancements often lead to public technological benefits.
    \item If Ben does not develop the idea, someone else will, potentially without ethical considerations.
\end{itemize}
}

\noindent Group A concluded by highlighting:
{\fontsize{11pt}{13pt}\selectfont
\begin{itemize}
\setlength{\itemsep}{4pt} 
    \item The risks of escalating military capabilities, especially in the current political climate.
    \item The company is not in the military sector, so Ben did not sign up for this work.
    \item If technological advancements are inevitable, Ben still has the moral right to abstain.
\end{itemize}
}

\section{Question 2}
\textbf{You are a computer engineer and you are assigned on the Manhattan Project. Do you think it is ethical to work on this project? (Group A No)}
\textit {Group A - No}
\textit {Group B - Yes}\\

\noindent Group A began by arguing:  
{\fontsize{11pt}{13pt}\selectfont  
\begin{itemize}  
\setlength{\itemsep}{4pt}   
    \item They would not contribute to the atomic bomb, even if it meant ending the war sooner, as they would still bear the moral burden of the deaths.  
    \item Japan was already weakened, and Germany had been defeated. Alternative measures could have ended the war.  
\end{itemize}  
}  

\noindent Group B countered:  
{\fontsize{11pt}{13pt}\selectfont  
\begin{itemize}  
\setlength{\itemsep}{4pt}   
    \item After the bomb’s creation, scientists established the Federation of American Scientists to regulate atomic energy and consider public interests.  
    \item Beyond warfare, the research led to significant scientific advancements.  
\end{itemize}  
}  

\noindent Group A responded:  
{\fontsize{11pt}{13pt}\selectfont  
\begin{itemize}  
\setlength{\itemsep}{4pt}   
    \item Nuclear energy could have been developed without the atomic bomb.  
    \item Weapons of mass destruction are inherently unethical and unjustifiable.  
\end{itemize}  
}  

\noindent Group B argued that from a utilitarian perspective, if the bomb ended the war and saved more lives than it took, then developing it was ethical. Additionally, from a rule-utilitarian standpoint, no established rule at the time deemed it unethical.\\

\noindent Group A countered that even under utilitarianism, the bomb was unjustifiable. It caused long-term harm to people and the environment, ultimately outweighing any perceived benefit.\\

\noindent Group B pointed out that, at the time, it was an arms race. If the U.S. had not developed the bomb, Germany might have. Furthermore, nuclear energy advancements were driven by wartime pressures and funding.\\ 

\noindent Group A noted:  
{\fontsize{11pt}{13pt}\selectfont  
\begin{itemize}  
\setlength{\itemsep}{4pt}   
    \item Scientists failed to consider long-term ethical responsibilities, a principle now emphasized in issues like climate change.  
    \item Unlike scientists in some other countries, those in the U.S. were not forced to participate—they had a choice.  
\end{itemize}  
}  

\noindent Group B stated that given the urgency of the war, U.S. efforts focused on national defense and preventing further attacks like Pearl Harbor.  

\noindent Group A emphasized:  
{\fontsize{11pt}{13pt}\selectfont  
\begin{itemize}  
\setlength{\itemsep}{4pt}   
    \item These were the only atomic bombs ever used in warfare. Extensive testing had already provided insights into their effects, making live deployment unnecessary.  
    \item The U.S. is often seen as the hero of the war, yet it remains the only country to have used nuclear weapons in combat, acting in its own self-interest like other nations.  
\end{itemize}  
}  

\noindent Group B concluded that while the bomb was dropped on a civilian area, the Japanese government had received prior warning to evacuate. 

\section{Question 3}
\textbf{Positive discrimination in favour of women is often proposed as a measure to address gender imbalance. What are the advantages and disadvantages of such measures? (Group A in agreement of positive discrimination)}
\textit {Group A - Yes}
\textit {Group B - No}\\
\noindent Group B began by arguing:  
{\fontsize{11pt}{13pt}\selectfont
\begin{itemize}
\setlength{\itemsep}{4pt}
    \item Positive discrimination can undermine an individual’s sense of achievement, as they may be recognized not for their merit but to fulfill a quota.  
    \item Opportunities should arise from personal merit and interest rather than being imposed by an external agenda.  
\end{itemize}
}

\noindent Group A countered that positive discrimination considers not only merit but also an individual’s potential. Since men and women have different emotional intelligence, they bring varied skills that may currently be undervalued.  

\noindent Group B argued that positive discrimination is a short-term solution, a mere “band-aid” that does not address the root problem. They believe it may worsen the situation by fostering a perception of tokenism.  

\noindent Group A responded that positive discrimination is necessary because women face unique challenges that men do not. Without it, natural biases—such as those arising from life experiences like maternity leave—would continue to create disparities.  

\noindent Group B asserted that positive discrimination can negatively impact hiring by prioritizing diversity over competence, potentially lowering workforce quality.  

\noindent Group A concluded that while positive discrimination may seem like a temporary fix, it can have long-term benefits by inspiring future generations of women through relatable role models.  

\noindent Group B argued that positive discrimination attempts to disrupt natural tendencies in career choices, as certain groups may naturally excel in specific fields—such as women in nursing and men in STEM.  

\noindent Group B also noted that positive discrimination can lead to reverse discrimination and that it tries to use a negative idea (discrimination) to achieve a positive outcome.

\section{Question 4}  
\textbf{You are the CEO of a local computer engineering company that employs 150 people. What measures and policies will you put in place to move towards a gender balance?}  

The strongest points discussed were:  
\begin{itemize}  
    \item Establish clear career advancement criteria to ensure equal opportunities and eliminate ambiguity in promotions and discrimination.  
    \item Implement blind recruitment, where potentially discriminatory information is concealed from hiring managers to ensure a fair selection process.  
    \item Include both male and female representatives in hiring panels to promote fairness and minimize bias.  
    \item Increase the visibility of women in underrepresented fields to inspire others. For example, feature women in STEM on company websites or host industry events showcasing female professionals.  
    \item Collaborate with educational institutions to challenge stigmas and encourage equal participation in all fields.  
    \item Accommodate diverse employee needs by offering flexible work hours, maternity leave, and childcare facilities.  
\end{itemize}  

\end{document}
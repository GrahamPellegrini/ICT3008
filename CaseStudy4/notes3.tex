\documentclass{article}
\usepackage{amsmath, graphicx, hyperref, geometry}
\geometry{a4paper, margin=1in}

\title{Case Study 2 - White-Collar Crime}
\author{Graham Pellegrini}
\date{\today}

\begin{document}

\maketitle

\section*{Notes}
\textbf{Case A - Espionage in Silicon Valley}
\begin{itemize}
    \item The Valley industry is a hotbed of espionage, due to the fast pace of technological innovation and the high stakes involved. With rapid outdates of 2 years.
    \item Computer chip desing is super expensive and illegal measures are often taken to obtain the designs of competitors.
    \item The size of such chips lead to ease of theft, as they can be easily hidden.
    \item Policies until recent have lacked to cover the complex nature of the industry.
    \item Individuals and contacts carry heavy value to enable the theft of such designs.
    \item Silicon Valley has high employee turnover rates due to the opportunities and competition, leading to a lack of loyalty.
\end{itemize}

\textbf{Case B - Price Fixing in the Electrical Equipment Industry}
\begin{itemize}
    \item Sherman Antitrust Act of 1890 prohibits price fixing and other anti-competitive practices.
    \item Conspirators argued that price fixing benefited the public by stabilizing prices and preventing price wars.
\end{itemize}

\textbf{Case C - The Case of the Corrupt Contractor}
\begin{itemize}
    \item Employers are able to avoid criminal penelaties by compensating the victims of their employees' crimes.
    \item For 3 decades information on asbestos fibers that filed the air in the factory was withheld from employees.
    \item There was erase of 1300 documents that would have shown the dangers of asbestos.
    \item Recent studies show that 38 percent of workers exposed to such asbestos have died from lung cancer.
    \item Manville Corporation filed for bankruptcy in 1982 due to the large number of lawsuits. But a court agreement was reached to pay out \$2.5 billion to the victims.
    \item Safty measures were also not taken in 1985 as cyanide gas was released into the air during silver recycling. As workers where not even given the saftey standards but rather useless paper face masks and cloth gloves.
\end{itemize}

\section*{Question 1}
\textbf{Employers have often been reluctant to prosecute employees who commit crimes against them. It is easier just to fire them, thereby avoiding court hassles and bad publicity. Given that companies need to make profits, is this reluctance to bring criminal charges against employees morally permissible and responsible? (Group A Yes vs Group B No debate)}

\textit{Yes}
\begin{itemize}
    \item By firing the employees, the company is abstaing from the invovlement with crimes committed by the employee. Showing that they do not support such actions.
    \item The company is able to avoid the bad publicity that would come with a court case.
    \item It is also normal for Individuals to abstain from court cases due to the time and money that would be spent on the case. Some issues do not feasibly warrant the time and money spent on a court case.
    \item Showing internal disciplinary actions to set an example to other employees.
\end{itemize}

\textit{No}
\begin{itemize}
    \item Lack of legal consequences may paint the wrong picture to other employees, showing that they can get away with such actions.
    \item It may seem more suspicious to the public if the company does not take legal action against the employee. As it may seem that the company is trying to hide the issue and it is part of a bigger plan.
    \item By taking action you are not only preventing internally employees from committing such crimes but also preventing the public as well.
    \item The company has a moral obligation to the public to prevent such crimes from happening again. Also presenting the company as a moral entity that is more valuable than the short term profits.
    \item The company has a moral obligation to the victims of the crime to bring the criminal to justice.
\end{itemize}

\section*{Question 2}
\textbf{Should White-Collar Crimes Be Treated More Lightly Than Violent Crimes? (Group A: Yes vs. Group B: No)}

\textit{Yes}
\begin{itemize}
    \item Unlike physical crimes, white-collar crimes do not involve physical harm to individuals or pose a voilent threat to society.
    \item The negative effects of white-collar crimes are financial losses on persons who have a knowledge of the risks involved.
    \item Prison sentences on white-collar criminals creates an unseatlement in industries where competition pushes such leagal boundaries and can have negative effects in advancing technology.
    \item Price fixing can be seen as a way to stabilize prices and prevent price wars, which can be beneficial to the public.
    \item Many of the commiters of white-collar crimes are not aware or believe that they are commiting a crime. As seen in the Electrical Equipment Industry case.
\end{itemize}

\textit{No}
\begin{itemize}
    \item White-collar crimes although not voilent can have bigger impacts on lifes as they lead to great financial losses and job losses which essentially degrade the quality of life and even can be linked to cases of sucide.
    \item Some white collar crimes scale to whole industries and can have a negative impact on the economy as a whole.
    \item They favor the rich and powerful and can lead to a lack of trust in the system.
    \item Taking a light stance on white-collar crimes can lead to a lack of deterrence and can lead to more crimes being committed.
    \item Legal system should be fair and equal to all crimes, regardless of the type of crime.
\end{itemize}

\section*{Question 3}
\textbf{Was Catanich in an Immoral Conflict of Interest by Moonlighting for Gopal? (Group A: Yes vs. Group B: No)}

\textit{Yes}
\begin{itemize}
    \item Catanich worked for both National Semiconductor and Gopal, which created a clear conflict.
    \item He accepted money in exchange for corporate secrets, prioritizing personal gain over loyalty to his employer.
    \item His actions directly contributed to the theft and sale of confidential information.
    \item Accepting money from a competitor undermined National Semiconductor’s ability to trust employees.
    \item Even if he did not initially intend to commit a crime, he participated in unethical conduct by sharing trade secrets.
\end{itemize}

\textit{No}
\begin{itemize}
    \item Many employees take on secondary jobs to supplement income. His additional work alone was not unethical.
    \item Gopal pressured him into leaking information by exploiting financial difficulties, making his involvement more circumstantial than intentional.
    \item If National Semiconductor paid fair salaries, employees would not be tempted to engage in external dealings.
    \item There is no evidence that Catanich intended to damage his employer—his actions were driven by desperation rather than malice.
    \item Gopal was the mastermind, actively recruiting insiders for espionage. Catanich was merely a pawn in the operation.
\end{itemize}

\section*{Question 4}
\textbf{Should Manville Executives Be Charged with Manslaughter, Murder, or No Crime? (Group A: Manslaughter or Murder vs. Group B: No Crime)}

\textit{Yes}
\begin{itemize}
    \item Manville executives had internal reports showing asbestos was deadly but still concealed the risks.
    \item They actively suppressed health studies and medical reports, misleading workers about the hazards.
    \item Asbestos exposure caused thousands of deaths. The sheer scale of suffering justifies severe legal consequences.
    \item Charging executives would deter future cases of corporate negligence.
    \item If Film Recovery Systems executives were convicted for knowingly endangering employees, Manville’s leadership should face similar charges.
\end{itemize}


\textit{No}
\begin{itemize}
    \item Unlike murder, there was no direct intention to kill anyone—only failure to act responsibly.
    \item Employees can sue for damages through civil lawsuits rather than criminal prosecution.
    \item At the time, asbestos dangers were less publicly understood. It is unfair to judge past decisions by modern health standards.
    \item Holding executives personally accountable for company actions sets a dangerous precedent.
    \item Manville has already paid billions in settlements, and its bankruptcy harmed the company itself.
\end{itemize}

\section*{Question 5}
\textbf{Should Companies Use Polygraph Tests to Detect Employee Theft? (Group A: Yes vs. Group B: No)}

\textit{Yes}
\begin{itemize}
    \item Regular testing deters employees from engaging in unethical activities.
    \item White-collar crime causes significant financial losses. Polygraph tests help identify risks early.
    \item Polygraphs are widely used in government and security sectors. Businesses should have similar tools.
    \item Honest employees have nothing to fear, and it ensures trust in the workforce.
    \item Stopping fraud early is cheaper than dealing with legal and financial consequences later.
\end {itemize}

\textit{No}
\begin{itemize}
    \item Testing all employees is excessive and creates a culture of distrust.
    \item They have a high false-positive rate, leading to wrongful accusations.
    \item Many countries ban polygraph tests in private businesses due to ethical concerns.
    \item Skilled individuals can learn to deceive lie detectors, making them ineffective.
    \item Constant suspicion creates a hostile work environment.
\end{itemize}

\end{document}




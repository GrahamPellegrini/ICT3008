\documentclass{article}
\usepackage{amsmath, graphicx, hyperref, geometry}
\geometry{a4paper, margin=1in}

\title{Case Study 2 - White-Collar Crime}
\author{Graham Pellegrini}
\date{\today}

\begin{document}

\maketitle

\section*{Notes}
\textbf{Case A - Espionage in Silicon Valley}
\begin{itemize}
    \item The Valley industry is a hotbed of espionage, due to the fast pace of technological innovation and the high stakes involved. With rapid outdates of 2 years.
    \item Computer chip desing is super expensive and illegal measures are often taken to obtain the designs of competitors.
    \item The size of such chips lead to ease of theft, as they can be easily hidden.
    \item Policies until recent have lacked to cover the complex nature of the industry.
    \item Individuals and contacts carry heavy value to enable the theft of such designs.
    \item Silicon Valley has high employee turnover rates due to the opportunities and competition, leading to a lack of loyalty.
\end{itemize}

\textbf{Case B - Price Fixing in the Electrical Equipment Industry}
\begin{itemize}
    \item Sherman Antitrust Act of 1890 prohibits price fixing and other anti-competitive practices.
    \item Conspirators argued that price fixing benefited the public by stabilizing prices and preventing price wars.
\end{itemize}

\textbf{Case C - The Case of the Corrupt Contractor}
\begin{itemize}
    \item Employers are able to avoid criminal penelaties by compensating the victims of their employees' crimes.
    \item For 3 decades information on asbestos fibers that filed the air in the factory was withheld from employees.
    \item There was erase of 1300 documents that would have shown the dangers of asbestos.
    \item Recent studies show that 38 percent of workers exposed to such asbestos have died from lung cancer.
    \item Manville Corporation filed for bankruptcy in 1982 due to the large number of lawsuits. But a court agreement was reached to pay out \$2.5 billion to the victims.
    \item Safty measures were also not taken in 1985 as cyanide gas was released into the air during silver recycling. As workers where not even given the saftey standards but rather useless paper face masks and cloth gloves.
\end{itemize}

\section*{Question 1}
\textbf{Employers have often been reluctant to prosecute employees who commit crimes against them. It is easier just to fire them, thereby avoiding court hassles and bad publicity. Given that companies need to make profits, is this reluctance to bring criminal charges against employees morally permissible and responsible? (Group A Yes vs Group B No debate)}

\textit{Yes}
\begin{itemize}
    \item By firing the employees, the company is abstaing from the invovlement with crimes committed by the employee. Showing that they do not support such actions.
    \item The company is able to avoid the bad publicity that would come with a court case.
    \item It is also normal for Individuals to abstain from court cases due to the time and money that would be spent on the case. Some issues do not feasibly warrant the time and money spent on a court case.
    \item Showing internal disciplinary actions to set an example to other employees.
\end{itemize}

\textit{No}
\begin{itemize}
    \item Lack of legal consequences may paint the wrong picture to other employees, showing that they can get away with such actions.
    \item It may seem more suspicious to the public if the company does not take legal action against the employee. As it may seem that the company is trying to hide the issue and it is part of a bigger plan.
    \item By taking action you are not only preventing internally employees from committing such crimes but also preventing the public as well.
    \item The company has a moral obligation to the public to prevent such crimes from happening again. Also presenting the company as a moral entity that is more valuable than the short term profits.
    \item The company has a moral obligation to the victims of the crime to bring the criminal to justice.
\end{itemize}

\section*{Question 2}
\textbf{Should White-Collar Crimes Be Treated More Lightly Than Violent Crimes? (Group A: Yes vs. Group B: No)}

\textit{Yes}
\begin{itemize}
    \item Unlike physical crimes, white-collar crimes do not involve physical harm to individuals or pose a voilent threat to society.
    \item The negative effects of white-collar crimes are financial losses on persons who have a knowledge of the risks involved.
    \item Prison sentences on white-collar criminals creates an unseatlement in industries where competition pushes such leagal boundaries and can have negative effects in their mtoive to innovate.
    \item Price fixing can be seen as a way to stabilize prices and prevent price wars, which can be beneficial to the public.
    \item Many of the commiters of white-collar crimes are not aware or believe that they are commiting a crime. As seen in the Electrical Equipment Industry case.
\end{itemize}

\textit{No}
\begin{itemize}
    \item White-collar crimes although not voilent can have bigger impacts on lifes as they lead to great financial losses and job losses which essentially degrade the quality of life and even can be linked to cases of sucide.
    \item Some white collar crimes scale to whole industries and can have a negative impact on the economy as a whole.
    \item They favor the rich and powerful and can lead to a lack of trust in the system.
    \item Taking a light stance on white-collar crimes can lead to a lack of deterrence and can lead to more crimes being committed.
    \item Legal system should be fair in weighing out the damage done by the crime rather than the type of crime.
\end{itemize}

\section*{Question 3}
\textbf{Was Catanich in an Immoral Conflict of Interest by Moonlighting for Gopal? (Group A: Yes vs. Group B: No)}

\textit{Yes}
\begin{itemize}
    \item Catanich worked under a contractual obligation to National Semiconductor. Moonlighting for Gopal was a breach of that contract, especially since the job he was doing was in direct competition with his employer.
    \item He accepted money in exchange for corporate secrets, prioritizing personal gain over loyalty to his employer.
    \item His actions where not just verbal but he also commited theft as he stole documents from his nearby supervisors desk. So he was aware of his actions but chose an easy way out of his financial difficulties.
    \item Even if it was common practice for such espionage in the industry, it does not make it right.
\end{itemize}

\textit{No}
\begin{itemize}
    \item Espionage is very common within the Silicon Valley industry and it is even more common and expected from lower ranking employees such as Catanich. The management failed to provide proper security measures to prevent such actions.
    \item From the case study we see how many victims fall into the hands of Gopal. He was a master manipulator and was able to exploit Catanich's financial difficulties to get him to leak information.
    \item The information that was leaked was ultimately sold to Intel, a company from which National Semiconductor also bought stolen information from. Showing how the industry played a dirsty game and employees such as Catanich where put in the crossfire.
    \item The high employee turnover rate in the industry shows that employees are not valued and are easily replaced. So Catanich's actions can be justified as the industry would not show the same loyalty to him.
\end{itemize}

\section*{Question 4}
\textbf{Should Manville Executives Be Charged with Manslaughter, Murder, or No Crime? (Group A: Manslaughter or Murder vs. Group B: No Crime)}

\textit{Yes}
\begin{itemize}
    \item Manville executives had internal reports showing asbestos was deadly but still concealed the risks. They actively suppressed health studies and medical reports, misleading workers about the hazards.
    \item No/Improper equipment was provided such as usless paper masks that could have mitigated such risks. So without providing mitigation the asbestos exposure caused thousands of deaths. The sheer scale of suffering justifies severe legal consequences.
    \item Charging executives would deter future cases of corporate negligence. As Manville was able to settle the court case, other instances of corporation manslaughter still occured such as the Film Recovery Systems with the cyanide.
    \item Furthermore, Film Recovery Systems faced such charges later on with the same case instances showing that Manville was not properly convicted.
\end{itemize}


\textit{No}
\begin{itemize}
    \item Appart from the immorality of putting profits over safety, the executives did not commit a crime under the law at the time. There where no worker safety laws or regulations in place at the time.
    \item Employees are able to sue the company for damages and the company has already paid out billions in settlements.
    \item Health standards at the time where not taken as stictly as they are now. Appart from employees companies such as cigarette companies, which 
    used far worse products and unreglated than todya, where also not held accountable for the health risks of their products posed to consumers.
\end{itemize}

\section*{Question 5}
\textbf{Should Companies Use Polygraph Tests to Detect Employee Theft? (Group A: Yes vs. Group B: No)}

\textit{Yes}
\begin{itemize}
    \item Polygraph tests put less uncertainties on the employer and company, enabling faster and more sound decisions making.
    \item They can help be a general deterent to wrong doings as deception and lies are harder to maintain. So it encourges a person to act rightfully in the first place.
    \item Honest employees have nothing to fear and should not cuase unseatlement since only the guilty would be caught or penailized.
    \item Stopping fraud early and identifying the guilty party can save the company money in the long run.
\end {itemize}

\textit{No}
\begin{itemize}
    \item Creates a culture of distrust and suspicion among employees, leading to a toxic work environment. Even the innocent employees will feel unease as they are being put in the same boat.
    \item Realiance on technology, such issues cannot be fully determined by a machine. If a false positive is given it can lead to the wrong person being accused.
    \item Many countries ban polygraph tests in private businesses and furthermore in court cases as they are unnaturally invasive and can be easily manipulated.
    \item Skilled individuals can learn to deceive lie detectors, making them ineffective. On the other hand, skilled individuals can also rig the test to give false positives.
    \item Lack of responsibility is provided if such machineds are used and something goes wrong. Who is to blame if a result does not match the truth?
    \item Other options such as professional investigators are more effective and give responsibility to a professional human rather than a machine.
\end{itemize}

\end{document}




\documentclass{article}
\usepackage{amsmath, graphicx, hyperref, geometry}
\geometry{a4paper, margin=1in}

\title{Case Study 2 - Product Liability}
\author{Graham Pellegrini}
\date{\today}

\begin{document}

\maketitle

\section*{Case A - Cadillac Chips}
General Motors (GM) installed chips in Cadillac models that increased emissions. As a result, they had to recall approximately 500,000 vehicles, costing them \$45 million in fines. The Environmental Protection Agency (EPA) claimed GM knowingly violated pollution standards, while GM argued that the EPA misinterpreted the regulations. The chips were modified to inject more fuel to prevent stalling, but this also led to increased emissions. GM was aware of the issue but continued using the chips. This recall was the first ever mandated for pollution reduction rather than safety concerns.

\textbf{Volkswagen}

In 2015, Volkswagen (VW) was caught using software in diesel engines to cheat emissions tests. The "defeat device" detected when the car was undergoing laboratory testing and adjusted performance to reduce nitrogen oxide (NOx) emissions. However, in real-world driving, emissions were up to 40 times the legal limit. The scandal affected 11 million vehicles worldwide and resulted in billions of dollars in fines, lawsuits, and criminal charges.

\subsection*{Question 1}
\textbf{Were the GM engineers responsible for the problems with the Cadillac Seville and Deville models? (Yes vs. No debate – if not, who was responsible?)}

\textit{Yes:}
\begin{itemize}
    \item The engineers knowingly used climate control testing insights to make a quick fix to the stalling problem. They took a shortcut that led to increased emissions, failing their ethical responsibility as engineers towards the environment and cheating the system.
    \item They could have opposed higher management and refused to use the chips. However, they knowingly exploited a regulatory loophole to avoid fines.
    \item Their failure to uphold ethical responsibilities affected the public, who suffered from increased emissions and later had to deal with the recall.
    \item Although no direct legislation prohibited this method, general consensus under the Clean Air Act was known, and the engineers could not claim ignorance.
    \item A proper solution was later implemented with new fuel calibration, proving that an ethical approach was possible from the beginning.
\end{itemize}

\textit{No:}
\begin{itemize}
    \item The engineers were not responsible; they were following orders from higher management to prevent vehicle stalling. They did not directly decide to manipulate emissions.
    \item Emission testing standards failed to detect these issues, placing some responsibility on the EPA and legislative bodies for inadequate regulations.
    \item The Clean Air Act did not have explicit regulations against the method used, so the engineers did not technically break any rules.
    \item The competitive nature of the automotive industry pressured engineers to deliver a market-competitive product. Their primary responsibility was to meet corporate objectives.
    \item Competing companies opposed GM’s methods due to their own competitive interests rather than ethical concerns.
    \item GM ultimately found a proper solution only after competitive pressures eased and appropriate regulations were enforced.
\end{itemize}

\subsection*{Question 2}
\textbf{Considering the 2015 Volkswagen emissions scandal, were the engineers responsible for ensuring compliance with emission standards? (Yes vs. No debate. If not, who was responsible? If yes, what should they have done?)}

\textit{Yes:}
\begin{itemize}
    \item While the engineers were not solely responsible, they had a duty to adhere to their professional code of conduct.
    \item The engineers specifically designed a system to detect test conditions and manipulate performance to lower emissions, making this case worse than GM’s as it was a deliberate attempt to cheat.
    \item Given the modern understanding of environmental impact, they should have recognized the severity of the emissions violation.
    \item Unlike the GM case, which was unprecedented, the VW engineers had prior examples and should have acted accordingly.
    \item The engineers could have opposed higher management and refused to develop a fraudulent system.
    \item Instead of designing a cheat system, they could have used those resources to develop a legitimate emissions reduction solution.
    \item A whistleblower eventually reported the issue, proving that some engineers were aware of the wrongdoing but did not act earlier.
\end{itemize}

\textit{No:}
\begin{itemize}
    \item VW’s executive management was ultimately responsible, as they made the final decision to implement the defeat device.
    \item The company chose to seek a quick fix rather than invest in an effective emissions reduction solution.
    \item Some engineers may have simply followed orders without fully understanding the extent of the fraud.
    \item Corporate culture may have pressured engineers into compliance, making it difficult to challenge unethical decisions.
\end{itemize}

\pagebreak
\section*{Case B - Pinto}
Ford’s Pinto had a dangerously positioned gas tank that caused fires in rear-end collisions. Ford was aware of the issue but rushed production to compete with foreign manufacturers. The gas tank placement led to leaks in low-speed crashes, despite safer alternatives being available. A cost-benefit analysis justified ignoring safety concerns, leading to criticism that Ford prioritized profits over lives.

\subsection*{Question 3}
\textbf{Was it appropriate for Ford to use cost-benefit analysis in deciding whether to implement a safety improvement? What responsibilities do engineers have in such situations? (Group A Yes vs. Group B No debate)}

\textit{Yes:}
\begin{itemize}
    \item Cost-benefit analysis is a standard tool for decision-making, used across industries, including food production and pharmaceuticals.
    \item At the time, car manufacturing was relatively new, and engineers were still learning about vehicle safety. Mistakes contributed to future safety improvements.
    \item Ford met all safety standards of the era and was not found guilty in court.
\end{itemize}

\textit{No:}
\begin{itemize}
    \item Human life should not be reduced to a financial equation. The analysis failed by not adequately valuing human safety.
    \item Engineers have a moral obligation to prioritize public safety over cost-cutting.
    \item Safer alternatives existed, and engineers could have pushed for their implementation.
    \item Even though Ford complied with regulations, they knew the gas tank placement was unsafe.
\end{itemize}

\pagebreak
\section*{Case C - TV Antenna}
A Houston TV station built a 1,000-ft antenna in Texas. A flawed lifting method caused the tower to collapse, killing seven riggers. The design firm refused last-minute modifications due to costs, and riggers implemented an unsafe workaround. Engineers declined to review riggers’ plans, citing liability concerns. Video footage captured the collapse, sparking debate over responsibility.

\subsection*{Question 4}
\textbf{Were the design engineers justified in refusing to review the lifting plans? Should they have recommended consulting a specialist? (Group A Yes vs. Group B No debate)}

\textit{Yes:}
\begin{itemize}
    \item The engineers were not responsible for the riggers' lifting plans and should not have been held liable for the collapse.
    \item Reviewing lifting plans was outside their expertise and scope of responsibility.
    \item There was no legal requirement for engineers to evaluate the riggers' methods.
\end{itemize}

\textit{No:}
\begin{itemize}
    \item The engineers had a moral obligation to ensure safety and should have advised consulting a specialist.
    \item They could have issued a formal warning stating the plans were unsafe, protecting themselves legally.
    \item As responsible professionals, they should have ensured safe handling of their design.
\end{itemize}

\end{document}

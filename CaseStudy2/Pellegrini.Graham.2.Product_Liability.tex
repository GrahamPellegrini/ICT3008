\documentclass[12pt]{article}
\usepackage{pdfpages}
\usepackage{xcolor}

\usepackage{setspace}
\onehalfspacing
\usepackage[a4paper, margin=1in]{geometry}

\title{Case Study 2 - Product Liability}
\author{Graham Pellegrini }
\date{\today}

\begin{document}

\maketitle

\section{Question 1}
\textbf{Were the GM engineers responsible for the problems with the Cadillac Seville and Deville models? (Yes vs No debate – if no, who was responsible?)}

\begin{itemize}
    \item [\textcolor{blue}{Yes}] GM engineers were aware that the cars exceeded EPA emission limits.
    \item [\textcolor{red}{Yes}] Regardless of management pressure, they had a professional duty and the opportunity to whistleblow.
    \item [\textcolor{red}{No}] The engineers followed company directives and did not directly decide to manipulate emissions.
    \item [\textcolor{red}{No}] Regulations did not explicitly prohibit non-climate-controlled testing, and the EPA failed to enforce stricter measures.
    \item [\textcolor{blue}{Yes}] The engineers used testing methods that did not reflect real-world conditions, intentionally misleading the EPA.
    \item [\textcolor{red}{No}] There was no clear legal obligation preventing the engineers from following instructions that where given to them.
    \item [\textcolor{blue}{Yes}] Other companies also bypassed regulations, and the form of cheating the system was common practice in the industry.
    \item [\textcolor{blue}{Yes}] The effort spent devising a workaround could have been used to develop a legitimate solution.
    \item [\textcolor{red}{Yes}] Engineers, given their expertise, have a responsibility to ensure products are safe and compliant with regulations.
    \item [\textcolor{blue}{No}] In corporate environments, directives may be unclear, and engineers might lack full knowledge of the situation.
    \item [\textcolor{red}{No}] Compliance is management's responsibility; engineers are expected to work within the provided framework.
    \item [\textcolor{blue}{Yes}] Engineers who implemented the bypass had the technical knowledge to understand the ethical and legal consequences.
    \item [\textcolor{red}{Yes}] They have a duty to protect both the environment and public safety.
    \item [\textcolor{blue}{No}] Engineers also have contractual obligations to their employer and must follow company directives.
\end{itemize}

\section{Question 2}
\textbf{Read about the recent Volkswagen emissions Scandal (2015). Were the engineers responsible when faced with the problem that their cars were not adhering to emission standards. (Yes vs No debate. If not who was responsible, if yes what should they have done?)}

\begin{itemize}
    \item [\textcolor{blue}{No}] Modern corporate culture prioritizes output, creating pressure to meet market demands.
    \item [\textcolor{red}{Yes}] The software was deliberately designed to cheat the system, and engineers were held responsible.
    \item [\textcolor{blue}{No}] Weak oversight by governing bodies responsible for auditing companies.
    \item [\textcolor{red}{No}] Engineers cannot be blamed, as their work is heavily driven by production demands.
    \item [\textcolor{red}{No}] Their role was simply to meet targets, regardless of the means.
    \item [\textcolor{blue}{Yes}] Regulators cannot be blamed since the software was explicitly designed to bypass regulations.
    \item [\textcolor{blue}{Yes}] The intentional nature of the cheat undermines the professionalism of the engineers involved.
    \item [\textcolor{red}{No}] Volkswagen had to protect its reputation, and managers put engineers in a difficult position.
    \item [\textcolor{red}{No}] Engineers had limited influence over design and decision-making within the company.
    \item [\textcolor{blue}{Yes}] They could have formed an internal group to address the ethical concerns of the software.
    \item [\textcolor{red}{Yes}] In hindsight, the issue damaged the company further; speaking up earlier could have prevented this.
    \item [\textcolor{blue}{Yes}] Engineers should have sought an alternative solution rather than compromising their integrity and reputation.
    \item [\textcolor{red}{No}] Other companies likely use similar systems, but their actions remain undisclosed.
    \item [\textcolor{blue}{Yes}] The fact that competitors engage in the same practice does not justify it.
    \item [\textcolor{red}{Yes}] Innovation is key in engineering, yet they opted for a workaround instead of a real solution.
    \item [\textcolor{blue}{No}] Engineers work within the constraints given by management, which limits their ability to innovate.
\end{itemize}

\section{Question 3}
\textbf{Is it appropriate to use data such as these in Ford’s decision regarding whether or not to make a safety improvement in its engineering design. What responsibilities do you think engineers have in situations like this?  (Group A Yes vs Group B No debate)}

\begin{itemize}
    \item [\textcolor{blue}{No}] Engineers, bound by their professional Code of Conduct and Ethics, cannot put a price on human life.
    \item [\textcolor{red}{Yes}] The company was conducting a risk assessment. It is not always possible to make a product 100\% safe.
    \item [\textcolor{red}{Yes}] Cost-benefit analysis is a standard practice across industries. For example, the food industry weighs the costs of using certain ingredients or chemicals. Since businesses aim to be profitable, the cost of implementing safety features must be considered.
    \item [\textcolor{blue}{Yes}] If every possible safety feature were included in a car, costs would be so high that consumers could not afford it. Additionally, competition allows consumers to choose safer alternatives from other manufacturers.
    \item [\textcolor{red}{No}] It is inhumane to put a price on suffering and death. Moreover, such values are questionable, especially if companies profit from them.
    \item [\textcolor{blue}{Yes}] Cost-benefit analysis is a common and necessary practice in business.
    \item [\textcolor{red}{No}] The company saved only \$11 per car by not implementing the safety feature—a small price to pay for consumer safety.
    \item [\textcolor{blue}{Yes}] The issue was identified during testing and made public. Ultimately, it was the consumer's choice to purchase the car.
    \item [\textcolor{red}{No}] Not all consumers conduct extensive research and may be unaware of the risks. Furthermore, car dealers prioritize sales and may not disclose potential dangers.
    \item [\textcolor{blue}{Yes}] The company did not violate any regulations. Stricter policies should have been in place to prevent such situations.
    \item [\textcolor{blue}{Yes}] Cost-benefit analysis is a standard business practice. However, such assessments are typically internal documents and not disclosed to the public.
\end{itemize}

\section{Question 4}
\textbf{Were the design engineers right to refuse to look at the lifting plans? Should they have recommended that the riggers consult a specialist?  (Group A Yes vs Group No debate)}

\begin{itemize}
    \item [\textcolor{blue}{Yes}] The engineers had the right to reject as it was not their contractual responsibility to see how to rig the tower properly.
    \item [\textcolor{red}{No}] Ethical flags are raised especially since we know that people died. The engineers should have at least reviewed the plans and given a warning.
    \item [\textcolor{blue}{Yes}] If there was no contractual obligation and the engineers where ordered specifically not to review the plans, due to previous similar incidents, they were right to refuse.
    \item [\textcolor{blue}{Yes}] The riggers where not limited to the engineers' expertise, and could have conntacted outside specialists.
    \item [\textcolor{red}{No}] The engineers where looking at the riggers drowning and did nothing to help.
    \item [\textcolor{red}{No}] The riggers where unbale to work as they intended as the top part of the tower was mentioned not to be dismanlted. But after the obligation not to dismanlted, no other instructions where given.
    \item  1:06:46
\end{itemize}

\end{document}
